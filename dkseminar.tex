\documentclass[aspectratio=169]{beamer}

\usetheme{Boadilla}
\usecolortheme{beaver}

\usepackage[utf8]{inputenc}
\usepackage[german]{babel}

% --- TITEL-DATEN ---
\title{\textsc{GIF}-Format \& LZW-Kompression}
\subtitle{Seminar Datenkompression -- WiSe 25/26}
\author[Kreso \& Braun]{Kristijan Kreso \and Mark Ian Braun}
\institute[Seminar Datenkompression]{Fachbereich Informatik}
\date{06. Februar 2026}

\setbeamertemplate{navigation symbols}{}

\newenvironment{slide}{\begin{frame}[fragile, environment=slide]}{\end{frame}}

\begin{document}

\begin{frame}
\titlepage
\end{frame}

\begin{frame}{Gliederung}
\tableofcontents
\end{frame}

% --- INHALT ---

% ==========================================
% SECTION: EINFÜHRUNG
% ==========================================
\section{Einführung}

\begin{frame}{Einführung}
    \begin{itemize}
        \item \textbf{Graphics Interchange Format} (.gif)
        \item wurde 1987 vom US-Unternehmen CompuServe entwickelt
        \item neben XBM ist GIF das erste Standardformat für Bilder im Web
        \item \textbf{damaliges Ziel:} Bilder so komprimieren, damit sie im Web schnell laden
    \end{itemize}
\end{frame}

\begin{frame}{Kernmerkmale}
    \begin{itemize}
        \item Bitmap-Grafikformat (Rastergrafiken)
        \item \textbf{Farbtiefe = 8 Bit} also max. 256 verschiedene Farben möglich
        \item ermöglicht \textit{statische Bilder} oder \textit{Animationen} (Kombination mehrerer Bilder)
        \item \textbf{verlustfreie} Kompression
        \item 2 Versionen: „87a“ und „89a“ (Erweiterung)
    \end{itemize}
\end{frame}

\begin{frame}{Historie}
    \begin{itemize}
        \item \textbf{Die Anfänge:} Erstes Standard-Bildformat im frühen Web (neben XBM).
        \item \textbf{Der Abstieg:} JPEG und PNG verdrängten GIF aufgrund besserer Farbtiefe/Transparenz.
        \item \textbf{Das Comeback (ab ca. 2010):}
        \begin{itemize}
            \item Revival durch Social Media, Foren (wie Reddit) und Messenger
        \end{itemize}
        \item \textbf{Heute:}
        \begin{itemize}
            \item integriert in Facebook, Instagram, WhatsApp
            \item Standardformat für Memes und Loops
            \item wird verwendet bei Logos
        \end{itemize}
    \end{itemize}
\end{frame}

% ==========================================
% SECTION: TECHNISCHE ASPEKTE
% ==========================================
\section{Technische Aspekte}

\begin{frame}{Überleitung: Von Pixeln zu Farbindizes}
    \begin{itemize}
        \item GIF ist ein Bitmap-basiertes Grafikformat
        \item Farben werden jedoch nicht als direkte RGB-Werte z.\,B.\ (255,0,0) gespeichert, sondern über \textbf{Farbindizes} aus einer \textbf{Farbpalette} referenziert
        \item Diese Farbtabelle enthält max. 256 Farben
        \item Die Bitmap wird in eine \textbf{lineare Sequenz} aus Farbindizes umgewandelt – gelesen zeilenweise von oben links nach unten rechts
        \item[$\Rightarrow$] \textbf{INPUT für den LZW-Algorithmus}
    \end{itemize}
\end{frame}

\begin{frame}{Technische Details}
    \textbf{Interner Aufbau:}
    \begin{itemize}
        \item \textbf{Header:} GIF87a bzw. GIF89a
        \item \textbf{Logical Screen Descriptor:} Bilddimensionen
        \item \textbf{Farbtabelle:} Definition der Paletten-Farben
        \item \textbf{Bilddaten:} LZW-komprimierte Index-Folge
    \end{itemize}
\end{frame}

\begin{frame}{LZW-Kompression}
    \begin{block}{Kernkonzept}
        GIF nutzt den \textbf{Lempel-Ziv-Welch (LZW)} Algorithmus zur verlustfreien Kompression.
    \end{block}
    \begin{itemize}
        \item \textbf{Das Wörterbuch:} Zentrale Komponente des Algos. Es speichert Sequenzen von Pixel-Indizes und weist ihnen eindeutige, kurze Codes zu.
        \item \textbf{Ziel:} Lange, sich wiederholende Muster in den Bilddaten durch diese kürzeren Codes aus dem Wörterbuch zu ersetzen, um Speicherplatz zu sparen.
        \item \textbf{Voraussetzung:} Encoder und Decoder starten mit dem identischen Grundalphabet (Farbindizes 0--255).
        \item Das Wörterbuch ist nicht statisch, sondern wird während des Prozesses (\textit{on-the-fly}) mit jedem neu entdeckten Muster erweitert.
    \end{itemize}
\end{frame}


\section{Demonstration}
\begin{frame}{Demonstration}
    \begin{center}
        \Large \textbf{Live-Vorführung unseres Visualisierungstools}

        \vspace{0.5cm}
        \small (Hier wechseln wir zur Browser-Ansicht)
    \end{center}
\end{frame}

\section{Bewertung der Kompressionsgüte}
% --- SLIDE 1 & 2: DAS QUIZ ---
\begin{frame}{Interaktives Quiz: Wo arbeitet LZW effizienter?}
    \centering
    Welches dieser Bilder lässt sich mit LZW stärker komprimieren?
\end{frame}

% --- SLIDE 3: DER MERKSATZ ---
\begin{frame}{Bewertung: Kompressionsgüte}
    \begin{block}{Merksatz: Die Effizienz von LZW}
        Der LZW-Algorithmus arbeitet am besten bei vielen sich wiederholenden Sequenzmustern. Je höher die \textbf{Redundanz}, desto höher die Kompressionsrate.
    \end{block}

    \vspace{0.5cm}

    \begin{itemize}
        \item \textbf{Stärken (Hohe Kompression):}
        \begin{itemize}
            \item Große einfarbige Flächen (Logos, Icons, Tabellen).
            \item Hier entstehen extrem lange Ketten identischer Indizes $\rightarrow$ kurze Codes im Wörterbuch decken riesige Bildbereiche ab
        \end{itemize}
        \item \textbf{Schwächen (Geringe Kompression):}
        \begin{itemize}
            \item Komplexe Verläufe, Rauschen oder detailreiche Fotos.
            \item viele unterschiedliche Farben
            \item \textbf{Worst-Case:} alle Pixel haben unterschiedliche Farbe
        \end{itemize}
    \end{itemize}
\end{frame}

% --- ABSCHLUSS (Nicht im Inhaltsverzeichnis) ---

\begin{frame}{Literaturverzeichnis}
    \begin{thebibliography}{10}
        \beamertemplatebookbibitems
        \bibitem{Author1} Name, Vorname (Jahr). \textit{Titel des Buches/Artikels}.
    \end{thebibliography}
\end{frame}

\begin{frame}
    \centering
    \Huge Vielen Dank für eure Aufmerksamkeit!
\end{frame}

\end{document}
